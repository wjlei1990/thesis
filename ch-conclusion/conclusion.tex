\newpage
\chapter{Conclusion}
\label{ch:conclusion}

In this work, we went through the improvements made for the GLAD-M25.
To overcome the I/O bottleneck, we introduced the ADIOS and ASDF, for storing 
model data and seismic data. To make the computation more efficient and
preformant, we rebuilt our tools using python and added new features into
those software.
Workflow management tools help us to automate the inversion and reduce possible
human-introduced errors. We proposed a new weighting schema to
mitigate the uneven distribution of seismic source and stations.
Our new model is generated using all the improvements we mentioned,
using 1,480 earthquakes and 10 conjugate gradient and L-BFGS iterations.

We presented our model together with many global and regional models. Through
those comparisons, we demonstrated that our model has superior resolution
not only for upper mantle structures, but also structures in lower mantle.

\section{Future Work}

There are definitely lots of works could be done in the near future.
The lower hanging fruit would be a deeper investigation in th model structures.
There are definitely lots of interesting geological structures emerging in the
GLAD-M25 model that would take us more time to look in. Vp/Vs ratio at
plumes regions is among one of them.
We also observed interesting subduction structures at South America,
Farallon slabs and Tibet.

There are also a few directions we would like to push for the third-gen GLAD
models. First, there are still more earthquakes out there. Our current earthquake
dataset covers most of the deep events from 1995 -- 2016 in the Global CMT Project.
However, there are still thousands of inter-mediate and shallow earthquakes no being
assimilated. We haven't added any events from 2017 and 2018 yet. We are planning
to push our dataset to 4,000 earthquakes in the near future, and new simulation
requires even more computing resource. We expect those simulations would be
performed Summit, the successor of Titan, at Oak Ridge National Lab. Our efforts 
in data format, software and workflow tools would enable us to handle a
even larger dataset without any issue.

We could also add more parameters into our model inversion.
Anelastic attenuation is an important factor that affects
the amplitude of waveforms, and it is also a very important
indicator of the water or chemical content in the mantle.
From our amplitude histograms, it is obvious that 
there are quite a lot of space for the current attenuation model to be improved.
Azimuth anisotropy is another important factor which has been taken
into account in GLAD-M25. It will also provide new insights
into those highly heterogeneous regions, such as plumes and subduction regions.

We also want to push the model resolution from 17 sec to a even lower period band,
9 sec. That will requires us
to have better source modelings. At global scale,
to ensure good signal-to-noise ratio, we usually picked events whose moment
magnitudes are between 5.5 to 7.2, and the largeest duration of source
time function would reaches ~17 sec.
When the simulation reaches such short period, source time function
becomes critical for the waveform fitting. Thus, to use all large earthquakes,
we need to have a better estimation rather than using
simple Gaussian type function.

At last, Source encoding could potentially bring new revolution into our inversion workflow.
With such technique, kernels could be calculated with one run(no gaps between forward and adjoint
simulation), so there is no need to save wavefield files, a procedure that stressed the file system
a lot and be our major challenges on HPC system push to higher resolution or larger dataset.
