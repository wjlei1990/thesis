\section{Comparison to Existing Formats}

Having yet another format induces more complexity and, potentially, noise into
the community using that type of data and the landscape of software able to
deal with it. ``Do we really need a new format?'' is thus a natural and
understandable question. This sections addresses why no single existing data
format in seismology is able to satisfy our needs and thus justifies the
introduction of the ASDF format.

We limit ourselves to detailing alternative waveform formats as we directly
incorporate the StationXML and QuakeML formats and no true alternative to
storing derived data or provenance is currently in existence.
A wide variety of different seismological data formats is used by researchers
world wide. We will discuss the most widely used ones, namely,
(Mini)SEED, SAC, and SEG Y/PH5.
Please see \cite{NMSOP-2} and \cite{Havskov} for additional information and
descriptions of more formats.


\subsection{MiniSEED}

The Standard for the Exchange of Earthquake Data (SEED) was developed in the
late 1980s and at least the data-only part (MiniSEED) continues to be in wide
use today, and will likely continue to be the dominant data streaming and
archival format for the foreseeable future. The ASDF format does not attempt to
replace it. Some of MiniSEED's features, such as the ability to build up large
data volumes by concatenating small and short pieces, are very well suited for
their use in data archives, where data is constantly streamed in.  While the
full SEED format can in theory store waveforms as well as station meta
information, the complexity of the format hinders that. It furthermore can only
properly store raw waveform recordings and no event information.  Additionally,
the dataless part of SEED, e.g., the part with the station information, sees
declining usage nowadays with that responsibility being taken over by
StationXML. MiniSEED, on the other hand, is more than capable of storing
arbitrarily large waveform volumes, but the file then contains no index of
what is in it, so one must always read the entire file to figure that out,
making large data volumes fairly impractical.  Additionally, the amount of
meta information in MiniSEED files is strongly limited, so one always needs
additional files to work with it.

Summing up, MiniSEED is a good data archival format for data centers,
streaming, and low-latency applications, but it is not well suited for the
later research and processing stages, where ASDF has significant
advantages.

\subsection{SAC}

The Seismic Analysis Code (SAC, \cite{HelffrichWookeyBastow201311})
introduced a new format named after its parent program, and is still in
widespread use today. This is likely due to two reasons: the popularity of the
SAC program itself and the relative simplicity of the format with a
number of header fields that can be adapted to different purposes.

The SAC format is well suited for many tasks, but ASDF offers
a number of advantages. The most obvious ones are the ability to store
multiple components ---including gaps and overlaps--- in a single file without
awkward workarounds, as well as the potential to create full datasets
incorporating all necessary meta information. ASDF is, for large
workflows, also more efficient, facilitates the storage of different data types ---
integers as well as floats --- and, with the help of HDF5 offers file
compression and check summing.

The combination of these factors results in ASDF being more suitable and
convenient for many workflows. Some, for example experiments with millions of
waveform files, are almost impossible without a more advanced seismological
data format.
In fact, part of the motivation for developing ASDF stems from the fact that
reading and writing SAC files for a large tomographic inversion practically
brings a huge parallel file system to its knees due to the very large number of
involved files.

\subsection{SEG Y and PH5}

\label{subsec:segy}

The SEG Y Data Exchange Format~\cite{SEGY} is one of many in the
family of data formats introduced and defined by the Society of Exploration
Geophysicists (SEG) Technical Standards Committee. Amongst these, it
is probably the most widely known and used. The more modern PH5~\cite{PH5}
format has a data model similar to SEG Y, but stores its
data in an HDF5 container. This eliminates some limitations of the
SEG Y format and facilitates more extensive meta information. It has
been developed as an archiving format for active source seismic experiments.
Typical workflows extract data from PH5 and save it as SEG Y,
which is used in the further stages.

Both on- and off-shore active source data is very structured, meaning that all
receivers generally have the same response and record for the same time span
with the same sampling rate.  Receivers are placed in lines and geo-referenced
by relative coordinates.  In contrast, passive source seismology is frequently
very unstructured, with different receiver types scattered across a
geographical region, and the meta information is fairly rich and detailed.

SEG Y and PH5 are well suited for active source experiments,
but it is difficult to adapt these formats for passive source seismologists to
suit their purposes.  Historically, SEG Y is essentially not used in
passive source seismology, and there is no reason to expect this will change
with PH5. The inverse is true as well, in that passive source
seismology tools are rarely used in active studies. A consequence is that the
current iteration of the ASDF format is not fully suitable for
exploration studies as it relies on certain formats and conventions. In the use
cases section we will show an example of how it can still be done.

The concept of seismic sources and receivers nonetheless holds true in both
active and passive source seismology. We have the hope that, in the future,
ASDF will be used as a standard for both. Active source
seismology currently lacks community accepted standards for sources and
receivers as is common in passive source seismology with formats like
QuakeML and StationXML.  Methods, ideas, and techniques are
frequently exchanged between these communities, and we encourage the
development of these missing standards.  A common data format would enable
greater sharing of tools, whole workflows, and most importantly human knowledge
and skill, greatly benefiting both sides.  The ASDF format is ready to
incorporate these aforementioned definitions.
