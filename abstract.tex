Using GLAD-M15 as our starting model, we conducted 10 quasi-Newton iterations to build
the second generation model, GLAD-M25. It pushes the resolution of global model to 17s
and increased the number of earthquakes to 1,480.
To achieve such goals, we worked from several
aspects to overcome various challenges. The Adaptable Seismic Data format, ASDF,
was designed to store the seismic data, to meet the requirements and standards
of modern data containers. GPU acceleration is introduced in the solver to facilitate
the forward and adjoint simulations. Workflow management tools was integrated to help
us automate the inversion workflow, reducing human interference and speed up the
iterations. To balance the uneven distribution of earthquakes and seismic stations,
we introduced the geographical weightings and applied them when constructing of misfit
functions, to facilitate the convergence rate.
Our model, GLAD-M25, was evaluated in various
ways, including the misfit and histogram comparisons.
We also used 360 earthquakes, as held-out
data, to assess the quality of the new model. All of the above showed our model has
significant improvements over our starting model, GLAD-M00 and GLAD-M15.
Our model is also show-cased with various global and region models.
We observed notable features
in our model, in Europe, Asia, North America and South America. Those comparisons
shows our model has unprecedented details compared to other global-scale models, reaching similar resolution as regional and Continental ones at upper mantle.
GLAD-M25 also shows prominent features in plumes and subduction zones, resolving deep structure in great details.
We believed such improvements in the interior of the Earth could 
provide us new insights in to the dynamics and evolution of the Earth.

