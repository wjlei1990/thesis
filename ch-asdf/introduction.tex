\section{Introduction}

\subsection{Motivation}

Seismology is, to a large extent, a science driven by observing, modeling, and
understanding data.  The process of making discoveries from data requires
simple, robust, and fast processing and analysis tools, empowering
seismologists to focus on actual science.  Modern seismological workflows
assimilate data on an unprecedented scale, and the need for efficient
processing tools is pressing.  In this context, the format in which data is
stored and exchanged plays a central role.  For example, passive seismic data
are commonly stored in such a way that each time series corresponds to a single
file on the file system.  The amount of I/O required to process and assimilate
data stored this way quickly becomes debilitating on modern HPC platforms.  As
another example, simulated seismograms depend on a large number of input
parameters, particular versions of modeling software, and specific run-time
execution commands.  A modern data format should strive for complete
reproducibility by keeping track of such data provenance.
The majority of existing seismic data formats were created in a more primitive
computing era, when no one could have foreseen the size, complexity, and
challenges that seismological datasets must accommodate today.
New seismological techniques, such as interferometry and adjoint tomography,
require access to very large computers, where I/O poses a major bottleneck and
data mining and feature extraction are challenging.

In this article we introduce a new data format ---the Adaptable Seismic Data
Format (ASDF)--- designed to meet these challenges.  We are fully
aware of the fact that the introduction of yet another seismic data format
should ideally be avoided.  However, we believe it to be justified because the
current state of the art is just not good enough. We further believe that the
advantages of the proposed format are significant enough to quickly outweigh
the initial difficulties of switching to a new format.
We identify five key issues that a new data format must resolve, namely:

\begin{itemize}

    \item \textbf{Efficiency:} Data storage is cheap, but data operations are
        increasingly becoming the limiting factor in modern scientific
        workflows. More efficient and better performing data processing and
        analysis tools are badly needed.

    \item \textbf{Data Organization:} Different types of data (waveforms,
        source \& receiver information, derived data products such as adjoint
        sources, receiver functions, and cross correlations) are needed to
        perform a variety of tasks. This results in ad hoc data organization
        and formats that are hard to maintain, integrate, reproduce, and
        exchange.

    \item \textbf{Data Exchange:} In order to exchange complex datasets, an
        open, well-defined, and community driven data format must be developed.

    \item \textbf{Reproducibility:} A critical aspect of science is the ability
        to reproduce results. Modern data formats should facilitate and
        encourage this.

    \item \textbf{Mining, Visualization, and Understanding of data:} As data
        volumes grow, more complex, new techniques to query and visualize large
        datasets are needed.
\end{itemize}

The ultimate goal is to empower seismologists to focus on actual science. This
is the time for the community to build an organized, high-performance, and
reproducible seismic data format for seismological research.
In order to facilitate integration of the new format into existing scientific
workflows and to demonstrate that this is not just an academic exercise, we
developed a Python library hooking ASDF into the ObsPy library~\cite{obspy2010},
which, as a hugely beneficial side-effect, also takes
%\cite{obspy2010}, which, as a hugely beneficial side-effect, also takes
care of any data format conversion issues, be it to, or from ASDF.  A C-based
ASDF library features an API for reading and writing ASDF files and includes
examples in both C and Fortran.  Embedding this library in the widely used
spectral-element waveform solver SPECFEM3D\_GLOBE
~\cite{KoTr02a, KoTr02b} made it gain native support for
ASDF-integrated workflows. To engage and educate the community, a wiki provides
demonstrations of the format and includes technical and non-technical
introductions for both users and developers.

\subsection{Scope}

The proposed Adaptable Seismic Data Format is designed to be an efficient,
self-describing data format for storing, processing, and exchanging
seismological data, including full provenance information.
It is intended to be used by researchers and analysts working with data, after
it has been recorded.
In contrast, it is not aspiring to replace the time proven MiniSEED format for
data archival, streaming, and low-latency applications. These use cases are
contrary to a comprehensive and self-describing data format and both can
probably not be achieved simultaneously.

ASDF is applicable to a large number of areas in seismology and
related sciences. Its use ranges from classical earthquake seismology to active
source datasets, ambient seismic noise studies, and GPS time series.
Furthermore, it is generic enough to accommodate any kind of derived or
auxiliary data that might accrue in the course of a research project.

\subsection{Benefits}

A well-defined format with the previously listed attributes directly results in
a number of advantages and applicable use cases. In this section we list a few
of these, in no particular order.

\begin{itemize}

    \item Seismological datasets usually contain waveform data as well as
        associated meta data, such as information about events and stations.
        All this data needs to be integrated and accessed concurrently, which
        requires a large amount of bookkeeping as datasets grow. Many tools are
        one-off scripts that cannot be reused for subsequent
        projects. Additionally, datasets become difficult to share with
        research groups that do not employ the same internal structure and data
        organization. Over the years, numerous groups have developed customized
        seismological data formats to work around these limitations. In
        contrast, ASDF is a well-defined format that can be used
        to store and exchange full seismological datasets, including all
        necessary meta information.

    \item It is oftentimes convenient to locally build up a database of
        preprocessed waveforms. A common example is storage of instrument
        corrected and bandpass filtered data. If a project continues for some
        years, it might ultimately no longer be known how exactly data were
        processed. The make up of the team may have changed, or perhaps the
        processing software had a bug that has been fixed in the meantime, and
        this may or may not have affected the data.  \emph{Provenance}, that
        is, the tracking and storing of the history of data, solves this
        particular problem, and ASDF accommodates that. Existing data
        formats do not (or only in a very limited manner) track the origin of
        data and what operations were performed on it due to limited and
        inflexible metadata allowances. ASDF is capable of storing the
        full provenance graph that resulted in a particular piece of waveform
        or other data.

    \item For the first time, ASDF accommodates proper storage and
        exchange of synthetic seismograms, including information about the
        numerical solver, earthquake parameters, the Earth model, and all other
        parameters influencing the final result. Waveform simulations at high
        frequencies and in physically plausible Earth models are extremely
        expensive computationally, so preserving and carefully documenting such
        simulations is of tremendous value.

    \item ASDF greatly reduces the number of files necessary for many
        tasks, because a single ASDF file can replace tens to hundreds
        of thousands of single waveform files. Beside raw performance and
        organizational benefits, this also facilitates workflows that run into
        hard file count quota limits on supercomputers. Note that
        ASDF can store data from very many receivers as well as
        arbitrarily long time series from only a single receiver and any
        combination in between.

    \item Importantly, ASDF offers efficient parallel I/O on modern
        clusters with the required hardware. This facilitates fully parallel
        data processing workflows that actually scale.

    \item ASDF offers optional and automatic lossless data
        compression, thereby reducing file size.

    \item Seismograms are certainly not the only type of data used in
        seismology. Other data types, including spectral estimations,
        cross correlations, adjoint sources, receiver functions, and so on,
        also benefit from organized and self-describing storage.

\end{itemize}


ASDF is intended as a container for all the various kinds of data
materializing in seismological research, including all required meta
information. Additionally, each piece of data should be able to describe itself
and what led to it. Having an organized and standard data container will, in
the long run, increase the speed and accuracy of seismic research, and provides
a medium for effectively communicating research results.
The rest of this chapter is structured as follows. We first provide an
overview of the layout of the format and justify some choices that needed to be
made. We then compare the ASDF format to existing data formats in use
in seismology, thereby further justifying its development. Finally, we showcase
a number of existing implementations, detail several use cases for the
ASDF format, and discuss future possibilities. The article is
intentionally light on technical details to focus on a high-level view. A
technical definition of the ASDF format can be found on line and in the electronic
supplements.
