\chapter{Introduction\label{ch:intro}}

The first construction of global tomographic models of the Earth
dates back to the late 1970s and early 1980s.
Around the same time, the theory of adjoint-state methods was
first applied in exploration seismology with the goal of
capturing the full physics of seismic wave propagation,
a process referred to as seismic full-waveform inversion (FWI).
Mainly due to computational challenges, it took until the late
2000s to see the first applications of adjoint-state methods
in regional- and continental-scale earthquake seismology.

The first generation global FWI model, GLAD-M15\cite{} was published in 2016
using 253 earthquakes and 15 conjugate-gradient iterations.

Over the past 6 years, we have been working on various aspects to improve
the model quality and publish the second-gen model, GLAD-M25.

The first thing is we did is increasing the dataset. We carefully picked
earthquakes from Global CMT project\cite{} and applied source inversion on those
earthquakes in 3D earth model. We carefully monitored how the mechanism changed
after source inversion and pick source with high quality data.
Our dataset covered most of the deep and inter-mediate depth
events in Global CMT categories. Even though they are usually small portion of
the overall earthquake dataset,  those deep events provide very clean deep phases
that sample the lower mantle structures.
The number of earthquakes was first pumped to 1,040, then increased to 1,480,
from GLAD-M15 to GLAD-M25.
It is almost a 6-fold of data increased and the dramatically increased dataset did brought
us lot of challenges.

It is very crucial to handle the I/Os with such a large dataset in a proper manner. 
SPECFEM3D GLOBE is a spectral finite element solver, whose performance is very
sensitive to the I/Os efficiency. With the 120min simulation at 17 sec resolution and 
undo-attenuation, the solver will generate 80GB of model files and 1TB of wavefield files.
Given the current size of 1,480 earthquakes, the total file size would reach 1.6 petabytes.
What is more critical is at the current performance of supercomputer Titan at Oak Ridge
National Lab, the 1.6 petabytes of files would be generated and saved to disk during
total of 6~hrs simulation. SPECFEM package used to use binary files to save model files and each
MPI task will operate on its own files, with each file can only save one single parameter.
Given the simulation usually runs on ~15,000 nodes, it can easily generate millions of I/Os
jam the I/O bus, bring the file system in its knees and slow down the simulations. Through
our collaboration with Oak Ridge National Lab, we integrate ADIOS in our solver and post-processing
tools. It greatly improved the I/O performance of our solver and enhance the stability of the
simulation. More details about ADIOS could be found in Appendix.~\ref{subsection:ADIOS}.

Another I/O challenge comes from the increase of seismic data. Unlike model and mesh data,
seismic data in general contains more components, including the earthquake source,
station information and time series data. With thousands of earthquakes, we have millions
of traces and station files. It is extremely
difficult to maintain the relationship and integrity given such a large number files. We used to manage
those files in directory based method, which is fragile and error-prone since directory
can be easily moved, copied or deleted. Another issue is the efficiency of data processing.
In our first-gen model, we used SAC to store waveform data and RESP files to store the instrument
response information. The shortcoming of such data formats are that they usually can only store 
data from one component from one station. To solve such issues, we developed the Adaptable
Seismic Data Format(ASDF) to store seismic data. One typical ASDF file will contain one
source(QuakeML) file, and all waveform and station(StationXML) associated with that event.
It also has the capability to store adjoint sources. ASDF will also provide APIs that takes
the user-defined data processing functions and dispatch computational task in parallel.
More details about ASDF could be found in Appendix.~\ref{chapter:asdf}.

It does take us a while to think about whether to re-write the data processing tools
into python. But it turns out that it is worth our time and efforts even though it does
take us some time to design and implement new python tools. First, re-writing in python
makes it possible to integrate with ASDF libraries, especially the parallel dispatch APIs.
When design data processing tools, we only need to focus on lower level, such as
operations on trace or stream level, without thinking about the parallel. Those logic is 
embedded in ASDF and thus parallel is much easier. Python also makes the code development
faster, relying on the community and open-source support, such as Numpy, Scipy and obspy.
Also, software management and deployment is much faster and easier, with the very powerful
python package management tools. More details for software practice could be found in 
section.~\ref{section:data_processing} and Appendix.~\ref{sec:software_practices}.

With so mange great components to be added in, it becomes very natural to introduce
workflow management tools to manage single components and connect them together. The data processing
workflow is used as an example to demonstrate the complex nature of the inversion workflow
in section.~\ref{section:workflow_management}. The workflow management tools can help us
to automate the inversion process, reducing time gaps between stages and also eliminate
human-introduced errors. It also increase the robustness of the system, by introducing
the job validation process after job exit from queue. It first checks the return code
of job status in queue, making sure there is no alarming errors. Then it will validate the
output file using the user-defined rules. If failure detected, it will resubmit the failed 
jobs in batch. It becomes very handy given that
hardware and file system failures are inevitable when processing such a large dataset.
To summarize, workflow management tools helped to integrate, stabilize, and automate
the entire FWI workflow.

Due to the uneven distribution of earthquakes and Seismic stations, the summation of
kernels will also be quite imbalanced in space. The problem becomes more severe with
the assimilation of regional seismic networks, such as US array. The dense regional
networks will have a strong, directional footprint on the gradients and model updates.
In GLAD-M15, we compensate such effect using multi-scale smoothing. But it didn't resolve
the unbalanced distribution. The geographical weighting was introduced, with a very simple
but robust algorithm to determine the weightings for sources and stations. With such 
weightings schemes, stations in dense regions will be assigned with smaller weighting values.
and to lower their contribution in the overall misfit function.
Simple 2D tests demonstrate that this geographical weighting scheme speeds up convergence of inversions
compared to traditional weighting schemes.
Examples of 3D Fr\'echet derivatives utilizing the geographical weighting scheme show much improved sensitivity in the deep mantle and in the poorly covered southern hemisphere.
Chapter.~\ref{ch:weighting} will discuss our weighting strategy in details.

Chapter.~\ref{ch:GLAD-M25} presents our second generation global tomographic model, GLAD-M25.
We presented our earthquakes dataset and seismic data selections. Source inversion
results are also presented and they are consistent with preceding researches.
Window selections for 1,480 earthquakes was also plotted, to give people a better
view about how good we are assimilating the different seismic phases.
We discuss the weighting schema and inversion strategy in the following sections.
Then, the GLAD-M25 model was evaluated in various ways, including an assessment of misfit reductions
in twelve measurement categories and
a statistical analysis of traveltime and amplitude anomalies.
A held-out database of 360 earthquakes was used to further interrogate the quality of our model,
showing similar misfit reductions and traveltime and amplitude anomalies as the actual inversion.
We conclude by showcasing GLAD-M25 together with many other global and regional models.
Regional upper-mantle horizontal slices are shown in the Europe, Asia, North and South America.
These comparisons illustrate that GLAD-M25 has unprecedented resolution, approaching that of regional models
in the upper mantle. We also showed vertical cross-sections comparisons in plumes and subduction regions.
In plumes regions, GLAD-M25 shows very consistent structures with other S wave models. In subduction
areas, GLAD-M25 has very similar structures as to other P-wave models but also shows clear slabs 
in the lower mantle.

In summary, Chapter \ref{ch:tools} gives an overview of the technical challenges we encountered
in the adjoint tomography workflow, covering data format, workflow components and management tools.
Chapter \ref{ch:weighting} discusses a geographical weighting scheme we introduced
in the inversion to balance the uneven distribution of earthquakes and seismographic stations.
Chapter \ref{ch:GLAD-M25} show-cased our new model, GLAD-M25, in great detail, covering input data,
inversion strategy and final results.
Appendix \ref{chapter:1Dmodel} and \ref{chapter:shanalysis} is supplement material for GLAD-M25,
providing spherical analysis to future evaluate the model. Appendix \ref{chapter:asdf} and
\ref{ch:exascale_tomography} is supplement material for chapter \ref{ch:tools}, as more details
for ASDF, ADIOS and workflow tools will be reveals in them. Appendix \ref{ch:software_resource}
provide links and DOI information to software we developed for the Global adjoint tomography project.