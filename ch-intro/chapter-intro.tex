\chapter{Introduction\label{ch:intro}}

Construction of global tomographic models of the Earth dates back to the late 1970s
and early 1980s. However, due to the computational chanllenges, it took more
than 20 years for the first adjoint-state models to be applied and shown successes
to regional and continental scale studies. The first generation of global adjoint 
model, GLAD-M15, was published in 2016 using 253 earthquakes.
During last 6 years, we have been working on extending the database and carry out
the second generation of the global model. The new model, GLAD-M25, used 1,480 earthquakes,
almost 6 fold of data compared last generation and ten quasi-Newton iterations starting
from GLAD-15. We believe the new model could provide researchers with new insights into
the interior and dynamics of the Earth.

To carry out iterations with such a large database and high demands for numerical
computations, there are several technique challeges we solved to achieve our goals.
Chapter 2 presents the Adaptable Seismic Data Format(ASDF). We discuss the motivation
to introduce a new modern data format, to host our seismic data. It was designed on
the purpose to faciliate the efficiency, integrity, and reproducibity
of seismic data that used in our inversion.

Chapter 3 gives a overview of the technique chanllenges we encounted in the workflow
of adjoint tomography, and our solution to them.
Our solver, SPECFEM3D GLOBE, takes the most expensive part of computation stages,
forward and adjoint simulations. It is accellerated using MPI and GPU, showing the
improvments by various benchmark tests. We also covered I/O problems, for both
mesher data and seismic data, by introductin ADIOS and ASDF to store different data.
The last part is workflow integration and automazition, using workflow management systems.

Chapter 4 discussed the weightings we introduced in the tomography, to balance
the uneven distribution of earthquakes and seicmic stations. We begun by demonstrating
a few current used weighting methods and point them drawbacks. We future presented
our simple-designed and robust weighting strategies. We conclude by conducting a 2D synthetic
test, showing our weighting strategy can indeed improve the convergence rate of global-scale
inversion.


Chapter 5 presented our second generation model, GLAD-M25. We went through the data and inverison
strategy over iterations. The model was evaluated by
various ways, including the misfit reduction and histogram change.
360 earthquakes were used as held-out data to future assess the quality of our model.
We conclude by showing case the GLAD-M25 model, along with many regional and global models. The details and improvment on resolutions is unseen in other global models and we belive such improvements could help researchers to get an better understanding of the earth at a global scale.
