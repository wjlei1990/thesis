This dissertation details the construction of a tomographic model of the Earth's crust and mantle
based on seismic full-waveform inversion (FWI).
The new model involves a database of 1,480 earthquakes in the magnitude range $5.5\le M_w \le 7.2$ recorded by more than 11,000 global
seismographic stations.
The approach utilizes 3D simulations of global seismic wave propagation with a shortest period of 17~s accelerated by Graphics Processing Units (GPUs),
requiring access to the world's largest and fastest supercomputers.
These simulations accommodate effects due to 3D anelastic crust \& mantle structure, topography \& bathymetry, the ocean load, ellipticity, rotation, and self-gravitation.
An adjoint-state method is used to calculate Fr\'echet derivatives in 3D anelastic Earth models
facilitated by a parsimonious storage algorithm.
The construction of the model,
named GLAD-M25, posed numerous challenges.
The Adaptable Seismic Data format (ASDF)
was designed to meet the requirements and standards
of modern data containers,
accommodating fast parallel I/O and full data provenance.
To manage the complex and fragile iterative global tomographic inversion,
workflow management tools were introduced to automate and harden the process.
To balance the uneven distribution of earthquakes and stations,
a geographical weighting scheme was introduced, thereby speeding up convergence in the
iterative quasi-Newton inversion algorithm.
The new model was evaluated by assessing misfit reductions and traveltime \& amplitude anomaly histograms in four period bands on three seismographic components.
Similar assessments were made for a held-out data set consisting of 360 earthquakes,
with results comparable to the actual inversion.
All of these evaluations show that model GLAD-M25 produces significant improvements in fit.
GLAD-M25 is compared with numerous other global and regional models,
and a wide variety of plumes and subduction zones are highlighted.
These comparisons demonstrate that GLAD-M25 has unprecedented resolution compared to other global-scale models, approaching that of regional- and continental-scale models of the upper mantle.
These improvements in resolution provide new insights into the dynamics and evolution of the Earth's interior.