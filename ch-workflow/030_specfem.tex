\section{Solver: \texttt{SPECFEM3D\_GLOBE}}
\label{sec:solver}

\subsection{Overview and programming approach}
%\red{Matthieu}

Our seismic inversion workflow includes many parts, each of them
implemented based on a different software package. The most computationally expensive
part involves the solver \texttt{SPECFEM3D\_GLOBE}, a spectral-element code capable of
modeling seismic wave propagation using fully unstructured hexahedral meshes of 3D
Earth models of essentially arbitrary complexity. Capabilities for its application
in adjoint tomography are extensively illustrated in~\cite{Peter2011}.
Table~\ref{tab:comp_storage_req_17s} outlines the importance of the solver
relatively to other parts of the workflow for a shortest period of 17\,s. As the
number of earthquakes grows, the importance of having a computationally
efficient solver to perform both forward and adjoint simulations becomes clear.

\begin{table}
\begin{center}
\tabletitle{Summary of computational requirements for a maximum resolution of 17~seconds,
using 180~minutes seismograms on OLCF infrastructure.}
\begin{tabular}{lccc}
%\toprule
       &   \tch{1 earthquake} & \tch{253 earthquakes}  & \tch{6,000 earthquakes}  \\
Shortest period: 17~s     &                      & \tch{1 iteration}      & \tch{1 iteration}       \\
Duration: 180~m &                      &                        &                   \\
\midrule%[\heavyrulewidth]
Solver (core hours) &  $\sim 11,520$   & $\sim 2,764,560$ & $\sim 69,120,000$ \\
(forward+adjoint)   & & & \\
%\midrule
Pre-processing    & $\sim 15$  & $\sim 3,795$     & $\sim 90,000$ \\
(core hours)      & & & \\
%\midrule
Post-processing  & $\sim 5,760$  & $\sim 5,760$     & $\sim 5,760$ \\
(core hours)     & & & \\
%\cmidrule(r){1-1}
%\bottomrule
\end{tabular}
%\captionsetup{singlelinecheck=off}
\label{tab:comp_storage_req_17s}
\end{center}
\end{table}

%\subsection{Programming approach}
At the top level, \texttt{SPECFEM3D\_GLOBE} is parallelized using MPI. The
parallelization extensively relies on the analytical decomposition of a
cubed-sphere mesh, splitting the mesh into partitions with close-to perfect load
balancing and assigning a single partition to a single MPI process. This coarse
domain decomposition parallelism may even include a finer level of parallelism
inside each domain.

To take advantage of architectures of modern processors, in particular multicore
CPUs and hardware accelerators such as GPUs, a finer level of parallelization is
added to each MPI process. For computations on multicore CPU clusters, partial
OpenMP support and vectorization has recently been added by Tsuboi et
al.~\cite{Tsuboi28022016}.

The \texttt{SPECFEM} software suite has been running on CUDA-enabled
GPUs since 2009~\cite{KoMiEr09}. It has continuously been adapted to take
advantage of advances in NVIDIA
GPUs~\cite{Komatitsch20107692, Rietmann2012}.

With the advent of heterogeneous supercomputers, our code base ---as for many
other applications--- has become increasingly complex. Due to a relatively large
user base and the variety of targeted systems, several code paths have to be
maintained. The most important issue is to ensure that similar capabilities are
available, regardless of the system. Another matter is to provide the user with
optimized software.

Our solution has been to use the BOAST~\cite{Cronsioe2013} transpiler to
generate both CUDA and OpenCL kernels. The performance of these BOAST-generated
kernels is very close to those implemented directly in CUDA, and provides code
optimized for various NVIDIA GPU generations.

On the one hand, because \texttt{SPECFEM3D\_GLOBE} strives to be a tool capable of solving
unexplored geophysical problems, most of it is written by geophysicists. On the
other hand, BOAST provides the programmer with a Ruby-based domain specific
language (DSL), an unfamiliar language for most of our developers. This implies
that while BOAST is equally capable of generating Fortran or C code, we maintain
the original Fortran code as the reference source for CPU execution.

In the future, and in particular for the next generation heterogeneous
supercomputers, it will be interesting to investigate
directive-based approaches. This is increasingly true with accelerator support
becoming more mature in OpenMP\,4. If performance is close to hand written code,
it will certainly be the solution of choice toward a portable, geophysicist-friendly code base.


%\subsubsection{Portability}

%\subsubsection{External Libraries}
