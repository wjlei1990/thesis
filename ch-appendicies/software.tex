\chapter{Software Resources\label{ch:implementation}}
\label{ch:software_resource}

Most of our software is open-source, hosted on either github (\url{https://github.com/}) or bitbucket (\url{https://bitbucket.org/}).
The repositories that I am the major contributor to are listed below. Urls for
repositories and DOI information is also listed with the software.\\

\noindent \textbf{pytomo3d} -- Low-level APIs for processing Stream objects\\
\url{https://github.com/computational-seismology/pytomo3d}\\
DOI: 10.5281/zenodo.55346\\

\noindent \textbf{pypaw} -- High-level APIs for processing ASDF files\\
\url{https://github.com/computational-seismology/pypaw}\\
DOI: 10.5281/zenodo.55346\\

\noindent \textbf{spaceweight} -- Geographical weighting calculations\\
\url{https://github.com/wjlei1990/spaceweight}\\
DOI: 10.5281/zenodo.56123\\

\noindent \textbf{pycmt3d} -- Source inversion software\\
\url{https://github.com/wjlei1990/pycmt3d}\\
DOI: 10.5281/zenodo.56124\\

\noindent \textbf{simpy} -- Seismic workflow management tools\\
\url{https://bitbucket.org/mpbl/simpy/src/master/}\\
(Matthieu Lefebvre and I are the main contributors to the simpy software.
Currently we provide limited access to this repository.)\\

Software that I contribute to is listed below.\\

\noindent \textbf{pyadjoint} -- APIs for measurements and adjoint sources\\
\url{https://github.com/computational-seismology/pyadjoint}\\

\noindent \textbf{pyflex} -- APIs for window selection\\
\url{https://github.com/computational-seismology/pyflex}\\

\noindent \textbf{pyasdf} -- ASDF I/O APIs\\
\url{https://github.com/wjlei1990/pyasdf}\\

Both pyflex and pyasdf are local forks for upstream repositories,
with customization for global adjoint tomography.
