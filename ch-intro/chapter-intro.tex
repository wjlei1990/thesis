\chapter{Introduction\label{ch:intro}}

The first construction of global tomographic models of the Earth dates back to the late 1970s
and early 1980s.
Around the same time,
the theory of adjoint-state methods was first applied in exploration seismology with the goal of capturing the full physics of seismic wave propagation,
a process referred to as seismic full-waveform inversion (FWI).
Mainly due to computational challenges,
it took until the late 2000s to see the first applications of adjoint-state methods in regional- and continental-scale earthquake seismology.

The first generation global FWI
model, GLAD-M15, was published in 2016 using a database of 253 earthquakes.
Over the past 6 years, we have been working on extending this database to determine
the second generation global model. The new model, GLAD-M25, used seismographic data from 1,480 earthquakes,
an almost 6 fold increase compared to the first generation model, and required ten quasi-Newton iterations, starting from GLAD-M15.
The new model will provide new insights into
the physics and chemistry of the Earth's interior.

To carry out tomographic iterations with such a large database and with such a high demand for numerical
computations, several technical challenges needed to be overcome to achieve our goals.
Chapter 2 presents the Adaptable Seismic Data Format(ASDF). We discuss the motivation
for the introduction of a new modern data format to host our seismic data. It was designed to provide efficiency, integrity, and reproducibility
of seismographic data.

Chapter 3 gives an overview of the technical challenges we encountered in the adjoint tomography workflow, and our solutions to these challenges.
Our solver, SPECFEM3D GLOBE, is the most expensive part of the compute stages of the workflow,
involving forward and adjoint simulations. It is accelerated using MPI and GPUs, as documented by various benchmarks. We also encountered I/O problems, for both
mesh files and seismic data.
These I/O problems were dealt with by using the ADIOS and ASDF libraries.
The introduction of workflow management tools helped to integrate, stabilize, and automate the entire FWI workflow.

Chapter 4 discusses a geographical weighting scheme we introduced in the inversion to balance
the uneven distribution of earthquakes and seismographic stations.
Simple 2D tests demonstrate that this geographical weighting scheme speeds up convergence of inversions
compared to traditional weighting schemes.
Examples of 3D Fr\'echet derivatives utilizing the geographical weighting scheme show much improved sensitivity in the deep mantle and in the poorly covered southern hemisphere.

Chapter 5 presents our second generation global tomographic model, GLAD-M25.
The model was evaluated in
various ways, including an assessment of misfit reductions in twelve measurement categories and
a statistical analysis of traveltime and amplitude anomalies.
A held-out database of 360 earthquakes was used to further interrogate the quality of our model,
showing similar misfit reductions and traveltime and amplitude anomalies as the actual inversion.
We conclude by showcasing GLAD-M25 together with many other global and regional models.
These comparisons illustrate that GLAD-M25 has unprecedented resolution, approaching that of regional models.
