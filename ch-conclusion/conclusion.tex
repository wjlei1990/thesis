\newpage
\chapter{Conclusion}
\label{ch:conclusion}

This thesis summarizes work that culminated in the construction of a global tomographic model,
called GLAD-M25, based on seismic Full Waveform Inversion (FWI).
The main phases of work conducted over the course of my graduate studies may be summarized as follows.
1) To overcome crippling I/O bottlenecks, we introduced a scalable I/O library (ADIOS) for reading and writing model data,
and an adaptable seismic data format (ASDF) for reading and writing seismographic data.
2) To make the computations more efficient and manageable,
we developed new Python toolkits to manage data assimilation and iterative model updates.
3) Workflow management tools were introduced to automate the inversion process and reduce possible
human-introduced errors.
4) We proposed a new weighting strategy to
mitigate the uneven distribution of seismic sources and stations.

Collectively, this work enabled the construction of global Earth model GLAD-M25,
using seismographic data from 1,480 earthquakes.
The new model was compared extensively with many existing global and regional models.
Based on these comparisons, we conclude that our model has superior resolution
not only in the shallow mantle, but also in the lower mantle.

\section{Future Work}

There are lots of opportunities for further research.
A deeper investigation of the new model is probably the lowest hanging fruit.
For example, the Vp/Vs ratio appears to be an excellent diagnostic for plumes.
Plumes are hotter and possibly wetter than surrounding mantle, which tends to lead to relatively high Vp/Vs ratios.
Since our model jointly constraints shear and compressional wavespeeds in the same frequency band,
we can reliably determine this ratio.
Our new model is fairly unique in this regard,
because very few other global studies can make a similar claim.
We also observe interesting subduction zone anomalies underneath South America,
possibly related to the subduction of the Fernandez and Nazca Ridges and the Inca Plateau.

For the third-generation model,
there is the opportunity to further increase the number of earthquakes to more than 4,000,
for example by adding events from 2017--2018.
These simulations will be performed on Titan's successor, a machine named `Summit' located at ORNL.
This is where our efforts to build scalable tools will really pay off.
Summit also enables us to reduce the shortest period of simulation from 17~s to 8.5~s,
which should help to further increase the resolution of our models.
However, this will require a more detailed representation of the seismic source than the current
centroid moment tensor point source solutions.
We are using earthquakes in the magnitude range 5.5--7.2, and the larger events have durations that approach the shortest period of simulation.
A first step would be the introduction of more complex source time functions,
perhaps with an element of directivity.

We should also add more model parameters to our inversion.
For example,
anelastic attenuation is an important factor that affects
the amplitude of waveforms~\cite{Zhuetal2013}.
It is considered to be an important
indicator of water content in the mantle.
From our amplitude anomaly histograms, it is obvious that 
there is quite a lot of room for improvement in the current 1D attenuation model.
Azimuthal anisotropy represents another important set of model parameters which has not been taken
into account in the construction of GLAD-M25.
Surface waves are known to be sensitive to azimuthal anisotropy in the upper mantle,
and such effects have been successfully accommodated in regional FWI~\cite{ZhuTromp2013}.

Finally,
new inversion strategies, such as source encoding,
could potentially revolutionize our inversion workflow.
With such techniques, kernels could be calculated in one single `super' simulation,
independent of the number of sources and receivers.
This not only saves computational cycles,
it also removes a huge amount of I/O and therefore pressure on the file system.
