\section*{Summary}

We present ASDF, the Adaptable Seismic Data Format, a modern and practical data
format for all branches of seismology and beyond.
The growing volume of freely available data coupled with ever expanding
computational power opens avenues to tackle larger and more complex problems.
Current bottlenecks include inefficient resource usage and insufficient data
organization. Properly scaling a problem requires the resolution of both these
challenges, and existing data formats are no longer up to the task.
ASDF stores any number of synthetic, processed, or unaltered waveforms in a
single file. A key improvement compared to existing formats is the inclusion of
comprehensive meta information, such as event or station information, in the
same file.  Additionally, it is also usable for any non-waveform data, for
example, cross correlations, adjoint sources, or receiver functions. Last but
not least, full provenance information can be stored alongside each item of
data, thereby enhancing reproducibility and accountability. Any dataset in our
proposed format is self-describing and can be readily exchanged with others,
facilitating collaboration.
The utilization of the HDF5 container format grants efficient and parallel
I/O operations, integrated compression algorithms, and check sums to guard
against data corruption. To not reinvent the wheel and to build upon past
developments, we use existing standards like QuakeML, StationXML, W3C PROV, and
HDF5 wherever feasible.
Usability and tool support is crucial for any new format to gain acceptance. We
developed mature C/Fortran and Python based APIs coupling ASDF to the widely
used SPECFEM3D\_GLOBE and ObsPy toolkits.
