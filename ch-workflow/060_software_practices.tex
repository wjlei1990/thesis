
\section{Software Practices}
\label{sec:software_practices}

The number and complexity of the software packages that we have developed, in order to be able
to perform exascale seismic inversions, have required us to adopt more rigorous
software practices.

Compared to professional software development teams, scientific software
developers face particular issues.
First, they are often a group of independent researchers that are in different
physical locations.
Second, there is often a large range in the level of programming
experience.
To address these issues, we implemented a simple
collaborative workflow based on modern software development techniques, such as
Agile
development and Continuous Integration/Continuous Development.

The two main goals of this workflow are to facilitate communication between
the developers, and to ensure that new software developments meet some agreed
upon quality criteria before being added to the common code base.
In practice, we have implemented this workflow using the tools provided by the
GitHub platform, and the automatic testing frameworks Buildbot, Jenkins and Travis CI.

%As illustrated in Figure \ref{fig:ReposBranches_all}, our collaborative
Our collaborative
software development practice
is organized around three Git repositories.
\begin{description}
\item[The central repository:] where the code is shared among developers, and
where users can download releases of the code.
\item[Forks of the central repository:] where each developer can post their
changes to
be tested by Buildbot, before being committed into the central
repository.
\item[Local clones:] where each developer builds his/her changes.
\end{description}
The first two repositories are hosted on the Computational Infrastructure for
Geodynamics (CIG) GitHub organization. The third repository is on the
developers desktop or laptop.

%\begin{figure}[tbh!]
%\centering
%\resizebox{5in}{!}{\input{ch-workflow/figures/ReposBranches_all.pdf_t}}
%\caption{There are three types of Git repositories: the central repository on
%  GitHub, the developer's fork on GitHub, and the developer's clone on their own
%  machine. The only way for developers to commit their changes to the central
%  repository is through a pull-request, which the code maintainers must accept.
%  Short tests are run before the maintainers accept the changes. Longer, more
%  extensive, tests are run on daily and weekly bases, as well as when new
%  developments are transferred from the \texttt{devel} to the \texttt{master}
%  branches of the central repository. \label{fig:ReposBranches_all}}
%\end{figure}

An essential part of our workflow is the differentiation between production and
development code: the production code is in a Git branch called
\texttt{master}, and the
development code is in a branch called \texttt{devel}.
The code in the \texttt{master} branch is intended for users that are only
interested in
running the code, while the \texttt{devel} branch is intended for code
developers. The changes to the code are first committed to the \texttt{devel}
branch.
Development code is transferred to the \texttt{master} branch only after
extensive
testing.

A fundamental rule of our workflow is that code changes can only be
committed to the \texttt{devel} branch of the central repository through a
pull-request. This provides us with two important features: first it allows us
to test the changes before they are committed to the central
repository, and, second, it sends an e-mail notification to the group of
developers. The notification to the developers is important because  it lets
them  review the changes before they are committed to the shared repository.

We have two distinct roles within our developer's community: \emph{code
maintainer}
and \emph{code developer}. The code maintainer role consists of accepting the
code changes proposed by the developers. The code maintainers have push/pull
(or read/write) permissions to the central repository, while the developers
only
have pull (or read only) permissions. In addition, code maintainers cannot
accept their
own source code changes.

Assuming that a developer already has a clone of the central repository on
his/her local machine, a typical workflow for committing new code developments
%to the central repository is as follows (see Figure \ref{fig:ReposBranches_all}).
to the central repository is as follows.
\begin{enumerate}
\item The developer pushes his/her changes to his/her fork on GitHub.
\item He/she opens a pull-request.
\item Opening a pull-request triggers automatic testing of the changes.
\item The maintainers and developers are notified of the results of the tests.
If the changes failed the tests, then the developer needs to fix the problems
and follow the steps 1 through 4, if they pass the test, go to step 5.
\item Before they can be committed in the central repository, the code needs to
be
reviewed by other developers.
\item The maintainers accept the changes and the new code is committed to the
  \texttt{devel} branch of the central repository.
\end{enumerate}

For this workflow to be successful, it is crucial to have a carefully designed
test suite. We use three types of tests for our codes.
\begin{description}
\item[Compilation tests:] they consist in looping through all the available
compiling options (e.g. OpenMP, MPI, CUDA) and using different compilers
(GCC and Intel compilers).
\item[Unit tests:] they tests individual functions by checking the output for
some
  predefined set of input parameters.
\item[Functional tests:] in our case, functional testing refers to the testing
of
  the a set of features of the code. Concretely, we run full examples for which
we
  compare the computed seismograms with some precomputed reference seismograms.
\end{description}

The compilation, unit, and some functional tests can be all done within 15~min
of opening the pull-request. These quick tests are the only ones that are done
before a pull-request is accepted. Other tests, that take longer to execute, are
run on daily and weekly bases. If these tests fail, then some changes need to be
reverted, but at least the changes are recent and there are few of them, so it is
easy to find what needs to to be fixed; we failed but we failed early.

In conclusion, our experience over the past three years has shown this
software development workflow to balance simplicity and effectiveness. Its
simplicity has made it easy to adopt for both experienced and new developers,
without hindering new developments. Its effectiveness at detecting problems
early has ensured the stability of our central repository.
In addition, by making the changes to the code more visible, this workflow
has improved the communication within our developer's community and enabled the
release of increasingly more sophisticated software.
