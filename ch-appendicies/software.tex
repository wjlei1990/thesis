\chapter{Software Resource\label{ch:implementation}}
\label{ch:software_resource}

Most of our software is open-source, hosted on github or bitbucket website.
The repocitory that I am major contributor are listed below. Urls for
repocitories and DOI information is also listed with the software.\\

\noindent \textbf{pytomo3d} -- low-level APIs for processing Stream object\\
\url{https://github.com/computational-seismology/pytomo3d}\\
DOI: 10.5281/zenodo.55346\\

\noindent \textbf{pypaw} -- high-level APIs for processing ASDF file\\
\url{https://github.com/computational-seismology/pypaw}\\
DOI: 10.5281/zenodo.55346\\

\noindent \textbf{spaceweight} -- geographical weighting calculation\\
\url{https://github.com/wjlei1990/spaceweight}\\
DOI: 10.5281/zenodo.56123\\

\noindent \textbf{pycmt3d} -- Source inversion software
\url{https://github.com/wjlei1990/pycmt3d}\\
DOI: 10.5281/zenodo.56124\\

\noindent \textbf{simpy} -- seismic workflow mangement tools\\
\url{https://bitbucket.org/mpbl/simpy/src/master/}\\
(Matthieu and I are the main contributors of the simpy oftware.
Currently we provide limited access to this repocitory.)\\

Software that I contribute to is listed below.\\

\noindent \textbf{pyadjoint} -- APIs for measurements and adjoint source
\url{https://github.com/computational-seismology/pyadjoint}

\noindent \textbf{pyflex} -- APIs for window selection
\url{https://github.com/computational-seismology/pyflex}

\noindent \textbf{pyasdf} -- ASDF I/O APIs
\url{https://github.com/wjlei1990/pyasdf}

Both pyflex and pyasdf are local forks for upstream repocitories,
with customizations for global tomography project added in.

