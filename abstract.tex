Using GLAD-M15 as our starting model, we conducted 10 quasi-Newton iterations to bring
the second generation model, GLAD-M25. It pushes the resolution of global model to 17s
and the number of earthquakes to 1,480. To achieve such goals, we work from several
aspects to overcome computational challenges. The Adaptable Seismic Data format,
was designed to store the seismic data, to meet the requirements and standards
of modern data containers. GPU acceleration is introduced in the solver to facilitate
the forward and adjoint simulations. Workflow management tools was used to help
us automate the inversion workflow, reducing human interference and speed up the
iterations. To balance the uneven distribution of earthquakes and seismic stations,
we introduced the geographical weightings to be applied to construction of misfit
functions and facilitate the convergence rate. Our model was evaluated in various
ways, including the misfit and histogram. We also used 360 earthquakes as held-out
data, to assess the quality of our new model. All the above showed our model has
significant improvements over the starting model. Our model is also demonstrated
and compared with various global and region models. We observed notable features
in our model, in Europe, Asia, North America and South America. The comparison
shows our model reaches unprecedented resolution for models at global scale, reach
similar resolution to regional and Continental ones at upper mantle. Our model also
shows prominent features in plumes and subduction zones, resolving deep structure
in great details. We believed such improvements in the interior of the Earth could 
provide us new insights in to the dynamics and evolution of the Earth.

